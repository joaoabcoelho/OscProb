\hyperlink{namespaceOscProb}{Osc\+Prob} is a small set of classes aimed at computing exact neutrino oscillation probabilities with a few different models.

\hyperlink{namespaceOscProb}{Osc\+Prob} contains a basic framework for computing neutrino oscillation probabilities. It is integrated into \href{https://root.cern.ch/}{\tt R\+O\+OT}, so that each class can be used as you would any R\+O\+OT class.

Available classes are\+:
\begin{DoxyItemize}
\item {\bfseries \hyperlink{classOscProb_1_1PremModel}{Prem\+Model}\+:} Used for determining neutrino paths through the earth
\item {\bfseries \hyperlink{classOscProb_1_1PMNS__Fast}{P\+M\+N\+S\+\_\+\+Fast}\+:} Standard 3-\/flavour oscillations
\item {\bfseries \hyperlink{classOscProb_1_1PMNS__Sterile}{P\+M\+N\+S\+\_\+\+Sterile}\+:} Oscillations with any number of neutrinos
\item {\bfseries \hyperlink{classOscProb_1_1PMNS__NSI}{P\+M\+N\+S\+\_\+\+N\+SI}\+:} Oscillations with 3 flavours including Non-\/\+Standard Interactions
\item {\bfseries \hyperlink{classOscProb_1_1PMNS__Deco}{P\+M\+N\+S\+\_\+\+Deco}\+:} Oscillations with 3 flavours including a simple decoherence model
\item {\bfseries \hyperlink{classOscProb_1_1PMNS__LIV}{P\+M\+N\+S\+\_\+\+L\+IV}\+:} Oscillations with 3 flavours including Lorentz Invariance Violations
\item {\bfseries \hyperlink{classOscProb_1_1PMNS__Decay}{P\+M\+N\+S\+\_\+\+Decay}\+:} Oscillations with 3 flavours including neutrino decays of the second and third neutrino mass states nu\+\_\+2 and nu\+\_\+3. \mbox{[}Requires external library Eigen3, see the instructions below.\mbox{]}
\end{DoxyItemize}

A few example macros on how to use \hyperlink{namespaceOscProb}{Osc\+Prob} are available in a tutorial directory.

\section*{Installing \hyperlink{namespaceOscProb}{Osc\+Prob}}

\hyperlink{namespaceOscProb}{Osc\+Prob} is very easy to install. The only requirements is to have R\+O\+OT installed with the G\+SL libraries.

{\bfseries N\+EW\+: Thanks to Jacek Holeczek, \hyperlink{namespaceOscProb}{Osc\+Prob} now also builds with R\+O\+OT 6!!}

In order to compile the \hyperlink{classOscProb_1_1PMNS__Decay}{P\+M\+N\+S\+\_\+\+Decay} class, it is necessary to donwload the external Eigen library. This library is added as a submodule. There are two options\+:
\begin{DoxyItemize}
\item During cloning\+: {\ttfamily git clone -\/-\/recurse-\/submodules \href{https://github.com/joaoabcoelho/OscProb.git}{\tt https\+://github.\+com/joaoabcoelho/\+Osc\+Prob.\+git}}
\item After clonning\+: {\ttfamily git submodule update -\/-\/init}
\end{DoxyItemize}

Once you have R\+O\+OT setup, simply do\+: 
\begin{DoxyCode}
cd OscProb
make
\end{DoxyCode}


A shared library will be produced\+: {\ttfamily lib\+Osc\+Prob.\+so}

This should take a few seconds and you are all set.

Just load the shared library in your R\+O\+OT macros with\+: 
\begin{DoxyCode}
gSystem->Load(\textcolor{stringliteral}{"/full/path/to/libOscProb.so"});
\end{DoxyCode}


Or use the {\ttfamily Load\+Osc\+Prob.\+C} macro (see below).

\section*{Tutorial}

In the directory Osc\+Prob/tutorial you will find a few macros with examples using \hyperlink{namespaceOscProb}{Osc\+Prob}.

Two macros are particularly useful\+:
\begin{DoxyItemize}
\item {\ttfamily simple\+Examples.\+C} \+: Contains some short pieces of code on how to perform different tasks.
\item {\ttfamily Make\+Oscillogram.\+C} \+: Runs a full example of how to plot an oscillogram with the P\+R\+EM model.
\end{DoxyItemize}

Additionally, these macros contain useful tools\+:
\begin{DoxyItemize}
\item {\ttfamily Load\+Osc\+Prob.\+C}\+: Searches for the \hyperlink{namespaceOscProb}{Osc\+Prob} library in your current directory, parent directory, or library path, and then loads it. It is called within the tutorial macros as a possible usage example.
\item {\ttfamily Set\+Nice\+Style.\+C}\+: Provides simple tools to make your plots look nicer. Feel free to use it anytime you\textquotesingle{}re making plots, even if you\textquotesingle{}re not running \hyperlink{namespaceOscProb}{Osc\+Prob}. This is completely independent of \hyperlink{namespaceOscProb}{Osc\+Prob}.
\end{DoxyItemize}

To run macros in compiled mode you will need to preload the \hyperlink{namespaceOscProb}{Osc\+Prob} library, e.\+g\+:


\begin{DoxyCode}
root -l LoadOscProb.C MakeOscillogram.C+
\end{DoxyCode}
 